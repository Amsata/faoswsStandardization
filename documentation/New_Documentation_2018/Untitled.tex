\documentclass[]{article}
\usepackage{lmodern}
\usepackage{amssymb,amsmath}
\usepackage{ifxetex,ifluatex}
\usepackage{fixltx2e} % provides \textsubscript
\ifnum 0\ifxetex 1\fi\ifluatex 1\fi=0 % if pdftex
  \usepackage[T1]{fontenc}
  \usepackage[utf8]{inputenc}
\else % if luatex or xelatex
  \ifxetex
    \usepackage{mathspec}
  \else
    \usepackage{fontspec}
  \fi
  \defaultfontfeatures{Ligatures=TeX,Scale=MatchLowercase}
\fi
% use upquote if available, for straight quotes in verbatim environments
\IfFileExists{upquote.sty}{\usepackage{upquote}}{}
% use microtype if available
\IfFileExists{microtype.sty}{%
\usepackage{microtype}
\UseMicrotypeSet[protrusion]{basicmath} % disable protrusion for tt fonts
}{}
\usepackage[margin=1in]{geometry}
\usepackage{hyperref}
\hypersetup{unicode=true,
            pdftitle={Standardization \& Balancing Commodity Tree dataset ~content and usage in the Food Blance Sheet framework},
            pdfauthor={Cristina Muschitiello~Food and Agriculture Organization of the United Nations},
            pdfkeywords={Commodity Tree, FBS, CPC, shares, extraction Rates, conversion factors},
            pdfborder={0 0 0},
            breaklinks=true}
\urlstyle{same}  % don't use monospace font for urls
\usepackage{graphicx,grffile}
\makeatletter
\def\maxwidth{\ifdim\Gin@nat@width>\linewidth\linewidth\else\Gin@nat@width\fi}
\def\maxheight{\ifdim\Gin@nat@height>\textheight\textheight\else\Gin@nat@height\fi}
\makeatother
% Scale images if necessary, so that they will not overflow the page
% margins by default, and it is still possible to overwrite the defaults
% using explicit options in \includegraphics[width, height, ...]{}
\setkeys{Gin}{width=\maxwidth,height=\maxheight,keepaspectratio}
\IfFileExists{parskip.sty}{%
\usepackage{parskip}
}{% else
\setlength{\parindent}{0pt}
\setlength{\parskip}{6pt plus 2pt minus 1pt}
}
\setlength{\emergencystretch}{3em}  % prevent overfull lines
\providecommand{\tightlist}{%
  \setlength{\itemsep}{0pt}\setlength{\parskip}{0pt}}
\setcounter{secnumdepth}{5}
% Redefines (sub)paragraphs to behave more like sections
\ifx\paragraph\undefined\else
\let\oldparagraph\paragraph
\renewcommand{\paragraph}[1]{\oldparagraph{#1}\mbox{}}
\fi
\ifx\subparagraph\undefined\else
\let\oldsubparagraph\subparagraph
\renewcommand{\subparagraph}[1]{\oldsubparagraph{#1}\mbox{}}
\fi

%%% Use protect on footnotes to avoid problems with footnotes in titles
\let\rmarkdownfootnote\footnote%
\def\footnote{\protect\rmarkdownfootnote}

%%% Change title format to be more compact
\usepackage{titling}

% Create subtitle command for use in maketitle
\newcommand{\subtitle}[1]{
  \posttitle{
    \begin{center}\large#1\end{center}
    }
}

\setlength{\droptitle}{-2em}
  \title{Standardization \& Balancing\\
\texttt{Commodity\ Tree} dataset\\
\hspace*{0.333em}content and usage in the Food Blance Sheet framework}
  \pretitle{\vspace{\droptitle}\centering\huge}
  \posttitle{\par}
  \author{Cristina Muschitiello~Food and Agriculture Organization of the United
Nations}
  \preauthor{\centering\large\emph}
  \postauthor{\par}
  \predate{\centering\large\emph}
  \postdate{\par}
  \date{7 June 2018}

\usepackage{lscape}
\usepackage{booktabs}
\usepackage{longtable}
\usepackage{array}
\usepackage{multirow}
\usepackage[table]{xcolor}
\usepackage{wrapfig}
\usepackage{float}
\usepackage{colortbl}
\usepackage{pdflscape}
\usepackage{tabu}
\usepackage{threeparttable}
\usepackage{threeparttablex}
\usepackage[normalem]{ulem}
\usepackage{makecell}

\usepackage{draftwatermark}
\usepackage{makeidx}
\makeindex
\usepackage{float}
\floatplacement{figure}{H}
\usepackage{amsmath}
\usepackage{amssymb}
\usepackage{amsthm}
\usepackage{mathtools}
\usepackage{caption}

\begin{document}
\maketitle
\begin{abstract}
This vignette provides a description of the \texttt{Commodity\ Treee}
dataset: it plays a key role in the Standardization and Balancing and
has been completely renewed in the new Food Balancing Framework. A
description is given in this document, functional to the Standardization
and Balancing Process.
\end{abstract}

{
\setcounter{tocdepth}{4}
\tableofcontents
}
\newpage

\listoftables

\listoffigures

\newpage

\subsection*{Disclaimer}\label{disclaimer}
\addcontentsline{toc}{subsection}{Disclaimer}

This Working Paper should not be reported as representing the official
view of the FAO. The views expressed in this Working Paper are those of
the author and do not necessarily represent those of the FAO or FAO
policy. Working Papers describe research in progress by the authors and
are published to elicit comments and to further discussion.

This paper is dynamically generated on \today{} and is subject to
changes and updates.

\section*{The Food Balance Sheet
Framework}\label{the-food-balance-sheet-framework}
\addcontentsline{toc}{section}{The Food Balance Sheet Framework}

\subsection*{Definitions}\label{definitions}
\addcontentsline{toc}{subsection}{Definitions}

A food balance sheet can be defined as an aggregated and analytical data
set that ``presents a comprehensive picture of the pattern of a
country's food supply during a specified reference period.''\footnote{For
  this definition and a more extended description of the motivation
  behind the development of FBS, see FAO, 2001, \emph{Food Balance
  Sheets: A Handbook}, available at:
  \url{http://www.fao.org/docrep/003/X9892E/X9892E00.HTM}. Accessed on
  19 January 2017.} FBS are presented as products accounts, where the
quantities allocated to all the sources of total supply must be equal to
the quantities allocated to all the sources of utilization. This balance
is compiled for every food item consumed within a country at primary
commodity equivalent basis, and all of the primary commodity equivalent
balances are then combined into a single overall FBS. FBSs are, then,
expressed in terms of per capita supply for each food item by dividing
by the country's population, with the per capita supplies being
expressed both in terms of quantity and, through the application of food
conversion factors, in terms of caloric value, protein, and fat content.
These per capita estimates of caloric value for individual food products
are then summed to obtain the total daily per capita Dietary Energy
Supply (DES) of a country.


\end{document}
